Problem 2: Compute the length of the astroid given by the parameterization $(\cos^3(t), \sin^3(t))$.

(Source: AoPS Calculus)

The length of the astroid is the integral of its speed with respect to $t$ from $t = 0$ to $t = 2\pi$.

We are given the parametric functions $x(t) = \cos^3(t)$ and $y(t) = \sin^3(t)$.

We can use the chain rule to compute the derivatives $\dfrac{dx}{dt}$ and $\dfrac{dy}{dt}$.

\begin{align*}
\frac{dx}{dt} = -3 \sin(t) \cos^2(t) \qquad \text{ and } \qquad \frac{dy}{dt} = 3 \sin^2(t) \cos(t)
\end{align*}

Now we can substitute these derivatives into the equation for length.

\begin{align*}
\int_{0}^{2\pi} \sqrt{\left(\frac{dx}{dt}\right)^2 + \left(\frac{dy}{dt}\right)^2} \, dt &= 4 \int_{0}^{\frac{\pi}{2}} \sqrt{\left(\frac{dx}{dt}\right)^2 + \left(\frac{dy}{dt}\right)^2} \, dt \\
&= 4 \int_{0}^{\frac{\pi}{2}} \sqrt{\left(\frac{dx}{dt}\right)^2 + \left(\frac{dy}{dt}\right)^2} \, dt \\
&= 4 \int_{0}^{\frac{\pi}{2}} \sqrt{\left(-3 \sin(t) \cos^2(t)\right)^2 + \left(3 \sin^2(t) \cos(t)\right)^2} \, dt \\
&= 4 \int_{0}^{\frac{\pi}{2}} \sqrt{9 \sin^2(t) \cos^4(t) + 9 \sin^4(t) \cos^2(t)} \, dt \\
&= 4 \int_{0}^{\frac{\pi}{2}} \sqrt{9 \sin^2(t) \cos^2(t) (\sin^2(t) + \cos^2(t))} \, dt \\
&= 4 \int_{0}^{\frac{\pi}{2}} \sqrt{9 \sin^2(t) \cos^2(t)} \, dt \\
&= 4 \int_{0}^{\frac{\pi}{2}} 3 \sin(t) \cos(t) \, dt \\
&= 4 \int_{0}^{\frac{\pi}{2}} 3 \cdot \frac{1}{2} \cdot \sin(2t) \, dt \\
&= 6 \int_{0}^{\frac{\pi}{2}} \sin(2t) \, dt \\
&= 6 \cdot -\frac{1}{2} \cos(2t) \, \Bigg|_{0}^{\frac{\pi}{2}} \\
&= -3 \cos(2t) \, \Bigg|_{0}^{\frac{\pi}{2}} \\
&= -3 \left(\cos(\pi) - \cos(0)\right) \\
&= -3 (-1 - 1) \\
&= -3 (-2) \\
&= \boxed{6}
\end{align*}

Thus the length of the astroid is $\boxed{6}$.
