Problem 7: Write parametric equations to describe the curve traced by the following motion: A point starting at the bottom edge of a bicycle wheel (with radius 30cm) that is rotating at 1 revolution per second, where the bicycle is moving forward at the rate implied by the rotation of the wheel. (Source: AoPS Calculus)

The position of the center of the wheel is given by

\begin{align*}
x_c(t) &= 0.6 \pi t \\
y_c(t) &= 0.3 \\
\end{align*}

The position of the point at the bottom edge is given by

\begin{align*}
x_p(t) &= .6 \pi t + 0.3 \cos \left(2 \pi t + \frac{\pi}{2}\right) \\
&= .6 \pi t - 0.3 \sin (2 \pi t) \\
y_p(t) &= 0.3 - 0.3 \sin \left(2 \pi t + \frac{\pi}{2}\right) \\
&= 0.3 - 0.3 \cos (2 \pi t)
\end{align*}

The position of the point at the bottom edge is given by the parameterization

$$ \boxed{\left( .6 \pi t - 0.3 \sin (2 \pi t), 0.3 - 0.3 \cos (2 \pi t) \right)} $$
