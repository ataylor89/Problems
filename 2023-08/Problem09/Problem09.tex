Problem 9: Write parametric equations to describe the curve traced by the following motion: A particle tracing a circle with center $(0, 0)$ and radius $2$, starting at $(2, 0)$ at time $t = 0$, moving clockwise with speed $\sqrt t$. (Source: AoPS Calculus)

We can figure out how much distance the particle travels in $t$ seconds.

\begin{align*}
d &= \int_0^t s(t) \, dt \\
&= \int_0^t \sqrt t \, dt \\
&= \frac{2}{3} t^{3/2} \Bigg|_0^t \\
&= \frac{2}{3} t^{3/2} \\
\end{align*}

The particle travels $\dfrac{2}{3} t^{3/2}$ distance in $t$ time.

Thus it makes $\dfrac{\dfrac{2}{3} t^{3/2}}{4\pi}$ revolutions in $t$ time, and travels $\dfrac{\dfrac{2}{3} t^{3/2}}{4\pi} \cdot 2\pi = \dfrac{1}{3} t^{3/2}$ radians in $t$ time.

Using this, we can write down the parametric equations.

\begin{align*}
x(t) &= 2 \cos \left( \frac{1}{3} t^{3/2} \right) \\
y(t) &= -2 \sin \left( \frac{1}{3} t^{3/2} \right) \\
\end{align*}

This gives a parameterization of

$$ \boxed{\left(2 \cos \left( \frac{1}{3} t^{3/2} \right), -2 \sin \left( \frac{1}{3} t^{3/2} \right)\right)} $$
