Problem 12: A polynomial of degree four with leading coefficient 1 and integer coefficients has two real zeros, both of which are integers. Which of the following can also be a zero of the polynomial? (Source: AMC 12)

$$ \frac{1 + i \sqrt{11}}{2}, \frac{1 + i}{2}, \frac{1}{2} + i, 1 + \frac{i}{2}, \frac{1 + i \sqrt{13}}{2}. $$

Let $f(x)$ be such a polynomial. Let $r$ and $s$ be the integer root of $f(x)$. Then

$$ f(x) = (x - r)(x - s)(x^2 + ax + b) $$

Expanding this gives

\begin{align*}
f(x) &= (x - r)(x - s)(x^2 + ax + b) \\
&= (x^2 - (r + s)x + rs)(x^2 + ax + b) \\
&= x^4 + ax^3 + bx^2 \\
& \quad - (r + s)x^3 - a(r + s)x^2 - b(r + s)x \\
& \quad + rsx^2 + arsx + brs \\
&= x^4 + (a - r - s)x^3 + (b - ar - as + rs)x^2 + (ars - br - bs)x + brs
\end{align*}

Now since the coefficients of $f(x)$ are integers, $a - r - s$ has to be an integer, so $a$ has to be an integer. Furthermore, $b - ar - as + rs$ has to be an integer, so $b$ has to be an integer.

Knowing that $b$ has to be an integer, we can rule out many possibilities. 

If $f(x)$ has nonreal roots, then the nonreal roots must form a conjugate pair. Their product must be an integer. The only possibility that meet this criteria is $\displaystyle \frac{1 + i \sqrt{11}}{2}$, since $\displaystyle \frac{1 + i \sqrt{11}}{2} \cdot \frac{1 - i \sqrt{11}}{2} = \frac{12}{4} = 3$ is an integer.

Thus $\boxed{\displaystyle \frac{1 + i \sqrt{11}}{2}}$ is the only complex number listed that can be a zero of the polynomial.
