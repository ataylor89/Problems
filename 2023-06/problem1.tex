Problem 1: If the six solutions of $x^6 = -64$ are written in the form $a + bi$, where $a$ and $b$ are real, then find the product of the solutions with $a > 0$. (Source: AHSME)

The sixth roots of unity are

$1, e^{2\pi i / 6}, e^{4 \pi i / 6}, e^{6 \pi i / 6}, e^{8 \pi i / 6}, e^{10 \pi i / 6}$

Now let $x^6 = -64$. We have $x = \sqrt[6]{-64} = 2\sqrt[6]{-1} = 2i$. 

We can get all six solutions if we multiply $2i$ by each of the sixth roots of unity.

$x = 2i, 2i e^{2\pi i / 6}, 2i e^{4 \pi i / 6}, 2i e^{6 \pi i / 6}, 2i e^{8 \pi i / 6}, 2i e^{10 \pi i / 6} $

Converting these solutions to rectangular form, we get

$ x_1 = 2i$
$ x_2 = 2i(\frac{1}{2} + \frac{\sqrt 3}{2}i) = -\sqrt{3} + i $
$ x_3 = 2i(-\frac{1}{2} + \frac{\sqrt 3}{2}i) = -\sqrt{3} - i $
$ x_4 = 2i(-1 + 0) = -2i $
$ x_5 = 2i(-\frac{1}{2} - \frac{\sqrt 3}{2}i) = \sqrt{3} - i $
$ x_6 = 2i(\frac{1}{2} - \frac{\sqrt 3}{2}i) = \sqrt{3} + i $

There are only two solutions with $a > 0$, where $a$ is the real part of the complex number. These solutions are $x_5 = \sqrt{3} - i$ and $x_6 = \sqrt{3} + i$. Their product is $\boxed{(\sqrt{3} - i)(\sqrt{3} + i) = 4}$.
