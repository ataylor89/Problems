April 1 2023

Problem 2: Prove that the polynomial
$$(a-x)^6 -3a(a-x)^5 +\frac{5}{2} a^2 (a-x)^4 -\frac{1}{2} a^4 (a-x)^2$$ takes only negative values for $0<x<a$.

(Source: Putnam 1941)

Solution

Let the polynomial $p(x)$ be defined as

$$p(x) = (a-x)^6 -3a(a-x)^5 +\frac{5}{2} a^2 (a-x)^4 -\frac{1}{2} a^4 (a-x)^2$$

We can start by factoring out $(a-x)^2$.

\begin{align*}
p(x) &= (a-x)^6 -3a(a-x)^5 +\frac{5}{2} a^2 (a-x)^4 -\frac{1}{2} a^4 (a-x)^2 \\
&= (a-x)^2 \Bigg[(a-x)^4 - 3a(a-x)^3 + \frac{5}{2} a^2 (a-x)^2 - \frac{1}{2} a^4\Bigg]
\end{align*}

Now let's expand the expression in brackets.

\begin{align*}
p(x) &= (a-x)^2 \Bigg[a^4 - 4a^3x + 6a^2x^2 - 4ax^3 + x^4 \\
& - 3a(a^3 - 3a^2x + 3ax^2 - x^3) \\
& + \frac{5}{2}a^2(a^2 - 2ax + x^2) - \frac{1}{2}a^4 \Bigg] \\
&= (a-x)^2 \Bigg[a^4 - 4a^3x + 6a^2x^2 - 4ax^3 + x^4 \\
& - 3a^4 + 9a^3x - 9a^2x^2 + 3ax^3 \\
& + \frac{5}{2}a^4 - 5a^3x + \frac{5}{2}a^2x^2 - \frac{1}{2}a^4 \Bigg] \\
&= (a-x)^2 (x^4 - ax^3 + \frac{1}{2}a^2x^2) \\
&= x^2 (a-x)^2 (x^2 - ax - \frac{1}{2}a^2)
\end{align*}

So far we have the result 

$$p(x) = x^2 (a-x)^2 (x^2 - ax - \frac{1}{2}a^2)$$

We can simplify this further. The quadratic formula allows us to get the roots of the quadratic.

\begin{align*}
x &= \frac{a \pm \sqrt{a^2 - 4 \left( \frac{-a^2}{2} \right)}}{2} \\
&= \frac{a \pm \sqrt{3a^2}}{2} \\
&= \frac{a \pm a\sqrt{3}}{2} \\
&= \frac{a(1 \pm \sqrt{3})}{2} \\
\end{align*}

Let $r_1 = \dfrac{a(1 + \sqrt{3})}{2}$ and $r_2 = \dfrac{a(1 - \sqrt{3})}{2}$.

We can factor $p(x)$ as such:

$$p(x) = x^2 (a-x)^2 (x - r_1) (x - r_2)$$

We want to show that $p(x)$ takes only negative values for $0 < x < a$.

We can verify that $r1 > a$ and $r2 < 0$.

\begin{align*}
0 < x < a &\implies x^2 > 0 \\ 
0 < x < a &\implies (a - x)^2 > 0 \\
0 < x < a &\implies (x - r_1) < 0 \\ 
0 < x < a &\implies (x - r_2) > 0
\end{align*}

When $0 < x < a$, the polynomial $p(x)$ becomes a product of three positive numbers and one negative number. Thus $p(x)$ only takes negative values when $0 < x < a$.
