April 2 2023

Problem 3: How many positive integers $N$ satisfy all of the following three conditions?

(i) $N$ is divisible by $2020$.\\
(ii) $N$ has at most $2020$ decimal digits.\\
(iii) The decimal digits of $N$ are a string of consecutive ones followed by a string of consecutive zeros.

(Source: Putnam 2020)

Solution:

Let $a_0 = 2020*55 = 111100$.

Let $a_n = a_{n-1}*10^4 + a_0$.

We can write out some terms of this sequence.

$$(a_n) = 111100, 1111111100, 11111111111100, ...$$

We want our sequence $(a_n)$ to be finite, since $N$ has at most 2020 decimal digits.

We can verify that $a_k$ has $4(k+1) + 2$ digits.

Let's find the maximum value of $k$.

\begin{align*}
4(k + 1) + 2 &= 2020 \\
4k + 6 &= 2020 \\
4k &= 2014 \\
k &= 503.5
\end{align*}

Our equations show that $a_{503}$ has 2018 digits.

Thus the sequence $(a_n)$ stops at $a_{503}$ and has a total of 504 terms.

Every term in $(a_n)$ satisfies the three conditions set out in the problem statement. Moreover, we can pad each term in $(a_n)$ with zeroes to create new numbers that satisfy the three conditions.

A term $a_n$ has $4n + 6$ digits.

A term $a_n$ can be used to generate $2020 - (4n + 6) + 1$ numbers that satisfy the three conditions set out in the problem statement.

Let $f(n) = 2020 - (4n + 6) + 1 = 2015 - 4n$.

The function $f(n)$ gives us the number of positive integers generated by $a_n$. 

The number of positive integers $N$ that satisfy all three conditions in the problem statement is given by the sum

\begin{align*}
\sum_{n = 0}^{503} f(n) &= \sum_{n = 0}^{503} 2015 - 4n \\
&= 504(2015 + 3)/2 \\
&= 504(2018)/2 \\
&= 504(1009) \\
&= 504,000 + 9 * 504 \\
&= 508,536
\end{align*}

There are $\boxed{508,536}$ positive integers that satisfy all three conditions.
